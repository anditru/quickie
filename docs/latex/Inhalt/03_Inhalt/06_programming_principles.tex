\chapter{Programming Principles}

\section{SOLID}

\section{GRASP}
GRASP steht für \enquote{General Responsibility Assignment Software Patterns/Principles}. Hierbei handelt es sich um neun Lösungsschemata für typische Fragestellungen in der Softwareentwicklung. Die Grundlage für GRASP bilden die beiden Prinzipien \enquote{Geringe Kopplung} und \enquote{Hohe Kohäsion}. Diese sollen in den beiden folgenden Abschnitten zuerst erläutert und anschließend aufgezeigt werden, inwiefern diese im vorliegenden Projekt zur Anwendung kamen.

\subsection{Geringe Kopplung}
Mit Kopplung wird in diesem Zusammenhang ein Maß für die Abhängigkeit einer Klasse von ihrer Umgebung bezeichnet. Das Ziel ist es, Code mit möglichst geringer Kopplung zu schreiben. Im vorliegenden Projekt wurde auf eine besonders geringe Kopplung zwischen der Applikationsschicht und Benutzeroberfläche Wert gelegt: Hier wird die geringste Stufe der Kopplung verwendet, bei der sich die beteiligten Kommunikationspartner nicht mehr kennen, nämlich eine rein eventbasierte Kommunikation durch die Adapterschicht. Auf diese Weise können auf möglichst einfache Weise weitere Schnittstellen, wie zum Beispiel ein HTTP-Interface, hinzugefügt werden.

Ein Beispiel für etwas stärkere Kopplung als bei der eventbasierten Kommunikation findet sich in den Serviceklassen der Adapterschicht. Müssen hier Daten persistiert werden, werden lediglich Methoden der im Domaincode definierten Interfaces für die Repositories aufgerufen und so eine direkte Kopplung an eine bestimmte Implementierung eines Repositories vermieden.

Ein Beispiel für starke Kopplung befindet sich in den Mapperklassen der Adapterschicht: Der \code{ProfileMapper} ist durch statische Methodenaufrufe stark an den \code{FoodMapper} und den \code{OpinionMapper} gekoppelt. Darüber hinaus werden der \code{FoodMapper} und der \code{OpinionMapper} auch nicht injected, sondern direkt im Konstruktor instanziiert, was die Kopplung zusätzlich erhöht.

\subsection{Hohe Kohäsion}
Die Kohäsion ist ein Maß für den inneren Zusammenhalt einer Klasse, zeigt also wie eng die Methoden und Attribute einer Klasse zusammenarbeiten. Ziel ist es, Klassen mit möglichst hoher Kohäsion zu schreiben.

Negativbeispiele, bei denen die Kohäsion in den Klassen gering ist, sind der \code{ProfileService} und der \code{RecipeService} in der Applikationsschicht. Hier betreffen die Methoden einer Klasse zwar nur eine Entity, allerdings implementieren die Methoden jeweils völlig unabhängige Use Cases, welche dennoch alle in einer Klasse zusammengefasst sind.

Beim dritten Service der Applikationsschicht, dem \code{MatchingService}, ist die Kohäsion hingegen hoch, da dieser lediglich einen einzigen Use-Case behandelt, nämlich das Finden ähnlicher Rezepte. Daher besitzt diese Klasse auch nur die öffentliche Methode \code{getMatchingRecipesFor}. Alle anderen Methoden sind privat und beinhalten Teilschritte des Matchingvorgangs.

Ein weiteres Beispiel für Klassen mit hoher Kohäsion sind die Mapperklassen der Adapterschicht, da sich jede der Klassen nur mit dem Mapping einer einzigen Entity bzw. eines Value Objects in verschiedene Formate befasst. Darüber hinaus befasst sich jede dieser Klassen nur mit einem einzigen Use-Case, dem Mapping. Damit gehören auch hier die Methoden einer Klasse semantisch zusammen.

\section{DRY}
DRY steht für \enquote{Don't Repeat Yourself} und ist der Name eines Programmierprinzips, welches besagt, dass redundanter Code vermieden werden sollte. Grund dafür ist, dass dadurch Wissen zentral festgehalten wird. Sollte das festgehaltene Wissen verändert werden müssen, so muss dieses folglich auch nur an einer Stelle geändert werden. Im Folgenden sollen Code-Beispiele diskutiert werden, welche Positiv- und Negativbeispiele für die Anwendung des DRY-Prinzips darstellen.

Ein Beispiel für das Einhalten des DRY-Prinzips ist in der Verifizierung der Parameter eines Callbacks zu sehen. Hier wird zentral in jeder Callback-Klasse durch die Methode \code{getRequiredParameters} definiert, welche Parameter benötigt werden. Das CLI Plugin nutzt diese Informationen, um die übergebenen Parameter in der Methode \code{ensureRequiredParameters} auf Vollständigkeit für den entsprechenden Callback zu prüfen sowie um die Benutzerhilfe in \code{constructOptions} zu generieren.

Ein Negativbeispiel für das Nichteinhalten des DRY-Prinzips ist in der Implementierung der zu persistierenden Entitäten zu sehen. Diese werden einmal zentral in der Domäne definiert, während es dann jedoch in der Adapterschicht noch die Interfaces \code{PersistentProfile} und \code{PersistentRecipe} gibt, welche durch Getter- und Setter-Methoden die notwendigen und zu persistierenden Eigenschaften der Entitäten definieren. Diese werden abschließend im \code{persistence} Plugin implementiert. Dadurch wird die Struktur einer Entität einmal in der Domäne und einmal in der Persistenz definiert, es gibt eine Dopplung der Informationen, welche Attribute zu einer Entität gehören. Diese Verletzung des DRY-Prinzips wurde begangen, um die Persistenz-Darstellung von der Arbeitsdarstellung zu entkoppeln, wodurch die Persistenz deutlich flexibler gestaltet werden kann.

Ein weiteres simpleres Negativbeispiel findet sich in Methoden des \code{JPAProfileRepository} und \code{JPARecipeRepository}. In diesen wird, falls ein \code{PersistentReicipe} oder \code{PersistentProfile} übergeben wird, immer zunächst geprüft, ob dieses eine Instanz eines \code{JPARecipe} beziehungsweise \code{JPAProfile} ist. Diese Überprüfung hätte in eine eigene Methode ausgelagert werden können, um Flüchtigkeitsfehler durch die Redundanz des Codes zu verhindern und die Wartung zu erleichtern.

Ein Beispiel für eine solche ausgelagerte Methode findet sich in der Klasse \code{CommandLineUI} mit der Methode \code{printHelp}. Diese Methode definiert zentral, wie die Hilfe ausgegeben werden soll und erfüllt somit das DRY-Prinzip.